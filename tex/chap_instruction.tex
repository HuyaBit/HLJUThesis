% !TeX root = ../Template.tex
% 本LaTeX模板的一般使用说明
\chapter{说明}

Again,大家好,这是黑大毕设\LaTeX{}模板(\CTeX{}-Based)---\HLJUThesis{}。

\HLJUThesis{}为黑大研究生学位论文模板,适用于理工类硕士。本\LaTeX{}模板代码fork的北航latex模板,格式参考自2017年版黑大《研究生学位论文撰写规范》(以下简称《规范》),最终成文格式需参考学院要求及打印方意见。本模板中大量内容和说明直接摘抄自《规范》,基本覆盖了论文内容和格式方面的要求。

文献著录BibTeX样式采用Haixing Hu开源的2005版参考文献著录BibTeX样式\href{https://github.com/Haixing-Hu/GBT7714-2005-BibTeX-Style}{GBT~7714-2005}及Zeping Lee开源的2015版参考文献著录BibTeX样式\href{https://github.com/zepinglee/gbt7714-bibtex-style}{GBT7714-2015},在此感谢两位的开源分享。请自行选用:\\

本模板已上传\href{https://github.com/HFbit/HLJUThesis}{GitHub}。

%-----------------------------
\section{封面问题}
感觉latex制作封面比较麻烦,就用了个简单的办法,具体方法如下\\
用Word生成pdf封面,将封面文件重命名成convercn.pdf,将生成的文件放到目录下的/convercn文件夹下即可,编译时会自动拼接论文与封面。

\section{宏包使用}

请将以下文件与此LaTeX文件放在同一目录中:

\begin{tabular}{ll}
 \verb|buaa.cls |                 & $\triangleright$ LaTeX宏模板文件 \\
 \verb|bst/GBT7714-2005.bst|      & $\triangleright$ 国标参考文献BibTeX样式文件2005 \\
 \verb|bst/GBT7714-2015.bst|      & $\triangleright$ 国标参考文献BibTeX样式文件2015 \\
 \verb|tex/*.tex|                 & $\triangleright$ 本模板样例中的独立章节\\
\end{tabular}\\

通过 \verb|\documentclass[| \verb|<thesis>,| \verb|<permission>,| \verb|<printtype>,| \verb|<ostype>,| \verb|<ctexbookoptions>| \verb|]{buaa}|载入宏包:
\begin{itemize}[leftmargin=3cm]
  \item[{\tt thesis} $\triangleright$]  论文类型(thesis),可选值:\\
    a) 学术硕士论文(\verb|master|)[缺省值]\\
    b) 专业硕士论文(\verb|professional|)\\
    c) 学术博士论文(\verb|doctor|)\\
    d) 专业博士论文(\verb|prodoctor|)
  \item[{\tt permission} $\triangleright$] 密级(permission),可选值: \\
    a) 公开(\verb|public|)[缺省值]\\
    b) 内部(\verb|privacy|)\\
    c) 秘密(\verb|secret|=\verb|secret3|)\\
    --- c.1) 秘密3年(\verb|secret3|)\\
    --- c.2) 秘密5年(\verb|secret5|)\\
    --- c.3) 秘密10年(\verb|secret10|)\\
    --- c.4) 秘密永久(\verb|secret*|)\\
    d) 机密(\verb|classified|=\verb|classified5|)\\
    --- d.1) 机密3年(\verb|classified3|)\\
    --- d.2) 机密5年(\verb|classified5|)\\
    --- d.3) 机密10年(\verb|classified10|)\\
    --- d.4) 机密永久(\verb|classified*|)\\
    e) 绝密(\verb|topsecret|=\verb|topsecret10|)\\
    --- e.1) 绝密3年(\verb|topsecret3|)\\
    --- e.2) 绝密5年(\verb|topsecret5|)\\
    --- e.3) 绝密10年(\verb|topsecret10|)\\
    --- e.4) 绝密永久(\verb|topsecret*|)
  \item[{\tt printtype} $\triangleright$] 打印设置(printtype),可选值: \\
    a) 单面打印(\verb|oneside|)[缺省值]\\
    b) 双面打印(\verb|twoside|)
  \item[{\tt ostype} $\triangleright$] 系统类型(printtype),可选值: \\
    a) Windows(\verb|win|)[缺省值]\\
    b) Linux(\verb|linux|)\\
    c) Mac(\verb|mac|)
  \item[{\tt ctexbookoptions} $\triangleright$] {\tt ctexbook}文档类支持的其他选项: \\
    使用{\tt ctexbookoptions}选项传递{\tt ctexbook}文档类支持的其他选项。
    例如,使用{\tt fontset=founder}选项启用方正字体以避免生僻字乱码的问题\footnote{需要系统安装方正字体。}。
\end{itemize}


\setlength{\hangindent}{4em}
模板已内嵌LaTeX工具包:\\
{\tt ifthen},{\tt etoolbox},{\tt titletoc},{\tt remreset},
{\tt geometry},{\tt fancyhdr},{\tt setspace},{\tt float},
{\tt graphicx},{\tt subfigure},{\tt epstopdf},{\tt array},{\tt enumitem},
{\tt booktabs},{\tt longtable},{\tt multirow},{\tt caption},
{\tt listings},{\tt algorithm2e},{\tt amsmath},{\tt amsthm},
{\tt hyperref},{\tt pifont},{\tt color},{\tt soul};\\
For Windows: {\tt times}, {\tt newtxmath};\\
For Linux: {\tt newtxtext}, {\tt newtxmath};\\
For Mac: {\tt times}, {\tt fontspec}。


模板已内嵌宏:\verb|\highlight{text}|(黄色高亮)。

请统一使用UTF-8编码。



%-----------------------------
\section{选项设置}

\begin{itemize}[leftmargin=3cm]
  \item[{\tt  $\backslash$refcolor} $\triangleright$]  开启/关闭引用编号颜色,包括参考文献,公式,图,表,算法等\\
  \texttt{on}:开启 [默认]\\
  \texttt{off}:关闭
  \item[{\tt $\backslash$beginright} $\triangleright$]  摘要和正文从右侧页开始\\
  \texttt{on}:开启 [默认]\\
  \texttt{off}:关闭
  \item[{\tt $\backslash$emptypageword} $\triangleright$]  空白页留字
  \item[{\tt $\backslash$Listfigtab} $\triangleright$]  是否使用图标清单目录\\
  \texttt{on}:开启 [默认]\\
  \texttt{off}:关闭
\end{itemize}


%-----------------------------
\section{章节撰写}
本模板支持以下标题级别标题级别:

\begin{tabular}{ll}
  \verb|\chapter{章}|              & $\triangleright$ 第一章 \\
  \verb|\chapter*{无章号章}|       & $\triangleright$ 无章号章 \\
  \verb|\chaptera{无章号有目录章}| & $\triangleright$ 无章号有目录章 \\
  \verb|\summary|                  & $\triangleright$ 总结\\
  \verb|\appendix|                 & $\triangleright$ 附录\\
  \verb|\achievement|              & $\triangleright$ 攻读学位期间取得的成果\\
  \verb|\acknowledgments|          & $\triangleright$ 致谢\\
  \verb|\biography|                & $\triangleright$ 作者简介\\
  \verb|\section{节}|              & $\triangleright$ 1.1 节\\
  \verb|\subsection{条}|           & $\triangleright$ 1.1.1 条\\
  \verb|\subsubsection{A}|         & $\triangleright$ 1.1.1.1 A\\
  \verb|\paragraph{a}|             & $\triangleright$ 1.1.1.1.1 a\\
  \verb|\subparagraph{a)}|         & $\triangleright$ 1.1.1.1.1.1 a)\\
\end{tabular}

%-----------------------------
\section{注意事项}
\begin{itemize}
  \item[$\triangleright$] \textit{中文斜体}将转换为楷体;
  \item[$\triangleright$] \textbf{中文粗体}在Windows(From WeiQM)和Mac(From CaiBW)下转换为黑体,Linux下正常(From QiaoJF);
  \item[$\triangleright$] \verb|\label{<text>}|中不能使用中文;
  \item[$\triangleright$] 浮动体与正文之间的距离是弹性的,根据内容调整,不太好控制;
  \item[$\triangleright$] 命令符与汉字之间请注意加空格以避免undefined错误(pdfLaTeX下好像一般不存在这个问题,主要在XeLaTeX编译环境下发生);
\end{itemize}

%-----------------------------
\section{ToDo}
\begin{itemize}
  \item[$\triangleright$] 数学环境的行间隔;
  \item[$\triangleright$] 参考文献的行间隔;
\end{itemize}

%-----------------------------
\section{意见及问题反馈}

\indent E-mail:2201610@hlju.edu.cn \\
\indent GitHub:\href{https://github.com/HuyaBit/HLJUThesis/issues}{https://github.com/HuyaBit/HLJUThesis/issues}
